\documentclass{article}
\usepackage{amsmath}
\usepackage{amssymb}
\usepackage{multicol}
\usepackage{amsfonts}
\usepackage{fancyhdr}
\usepackage{hyperref}
\usepackage{tabularx}
\usepackage[margin=1in]{geometry}
\usepackage{listings}
\usepackage{color}

\definecolor{dkgreen}{rgb}{0,0.6,0}
\definecolor{gray}{rgb}{0.5,0.5,0.5}
\definecolor{mauve}{rgb}{0.58,0,0.82}

\lstset{frame=tb,
  language=C,
  aboveskip=3mm,
  belowskip=3mm,
  showstringspaces=false,
  columns=flexible,
  basicstyle={\small\ttfamily},
  numbers=none,
  numberstyle=\tiny\color{gray},
  keywordstyle=\color{blue},
  commentstyle=\color{dkgreen},
  stringstyle=\color{mauve},
  breaklines=true,
  breakatwhitespace=true,
  tabsize=3
}
\begin{document}
\begin{flushright}
  Tim Tanasse\\
  CSCD 240\\
  \today \\
  Notes \# \# \\
\end{flushright}
\section*{C Conditionals and I/O}
\subsection*{If Statements and Loop Statements}
\begin{itemize}
  \item Literally the same usage as Java.
  \item Check demo1 and demo2 for examples.
  \item Switch demo1 and demo2
\end{itemize}
\subsection*{Basic I/O}
\begin{itemize}
  \item int getchar(void)
  \item this function reads the next avaiable character from the keyboard and returns it as an integer
  \item this function reads only one character, so frequently used in loops
  \item Be very careful with newline in buffer, as it is not read by getchar use while loops to empty buffer with $!= '\backslash n'$
  \item the char *gets(char *s) function reads a line from stin into the buffer pointed to by s until either a terminating newline or an eof.
  \item the int puts(const char *s) function writes the string s and a trailing newline to stdout
  \item the return value is non-negative if functioning, or EOF if error.
  \item the \textbf{int scanf(const char *format,...)} function reads input from the standard input stream stdin and scans that input according to format provided
  \begin{itemize}
    \item It should be noted that \textbf{scanf()} expects input in the same format as you provided in \%s and \%d, which means you have to provide valid input like "string integer"
    \item While reading a string \textbf{scanf()} stops reading as soon as it encounters a space, so "this is a test" are 4 strings for \textbf{scanf()}
    \item \textbf{scanf()} doesn't include newline feeds into the resulting string (VERY IMPORTANT)
  \end{itemize}
  \item the \textbf{ int printf(const char *format,...)} function writes output to the standard output stream stdout and produces output accourding to the format provided.
\end{itemize}
\end{document}
