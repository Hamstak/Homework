\documentclass{article}
\usepackage{amsmath}
\usepackage{amssymb}
\usepackage{multicol}
\usepackage{amsfonts}
\usepackage{fancyhdr}
\usepackage{hyperref}
\usepackage{tabularx}
\usepackage[margin=1in]{geometry}
\usepackage{listings}
\usepackage{color}

\definecolor{dkgreen}{rgb}{0,0.6,0}
\definecolor{gray}{rgb}{0.5,0.5,0.5}
\definecolor{mauve}{rgb}{0.58,0,0.82}

\lstset{frame=tb,
  language=C,
  aboveskip=3mm,
  belowskip=3mm,
  showstringspaces=false,
  columns=flexible,
  basicstyle={\small\ttfamily},
  numbers=none,
  numberstyle=\tiny\color{gray},
  keywordstyle=\color{blue},
  commentstyle=\color{dkgreen},
  stringstyle=\color{mauve},
  breaklines=true,
  breakatwhitespace=true,
  tabsize=3
}
\begin{document}
\begin{flushright}
  Tim Tanasse\\
  CSCD 240\\
  \today \\
  Notes \# \# \\
\end{flushright}
\section*{}
\begin{lstlisting}
    int **pptr = grades;
    printf("-1: pptr= %p\n", pptr);
    printf("-1: &pptr[0] = %p\n", &pptr[0] );
    printf("-1: pptr+1= %p\n", pptr + 1);
    printf("-1: pptr+2= %p\n", pptr + 2);
\end{lstlisting}
\textbf{Explanation: }the contents of \textbf{pptr} and the result of \text{\&pptr[0]} are the same, because pptr contains the address of \textbf{pptr[0]}: 0x7FB322403930. \textbf{pptr} points to integer pointers, which have a size of 8. Therefore the address of \textbf{pptr + 1} has an address of 0x7FB322403938, and \textbf{pptr + 2} has an address of 0x7FB322403940.
\begin{lstlisting}
    printf("0: pptr[0]= %p\n", pptr[0]);
    printf("0: *pptr= %p\n", *pptr);
    printf("0: &pptr[0][0]= %p\n", &pptr[0][0]);
\end{lstlisting}
\textbf{Explanation: }The contents of \text{pptr[0]} is the same as \textbf{grades[0]} and has an address of 0x7FB3224000E0. the dereferenced \textbf{pptr} and \text{pptr[0]} have the same value, which was recorded previously. The same is true of the next expression \textbf{pptr[0][0]} which is equivalent of \textbf{\&(*(*(pptr + 0) + 0))}. This translates similarly for \textbf{pptr[0]}, which is the same as \textbf{*(pptr + 0)} which is the same as \textbf{*pptr}
\begin{lstlisting}
    printf("1: pptr[1]= %p\n", pptr[1]);
    printf("1: *(pptr + 1)= %p\n", *(pptr + 1));
    printf("1: &pptr[1][0] = %p\n", &pptr[1][0] );
\end{lstlisting}
\textbf{Explanation: } The contents of \textbf{pptr[1]} is a memory address that is not contiguous of the previous addresses, that is, the contents of \textbf{pptr[1]} cannot be known by knowing the contents of \text{*pptr}, unlike the rest of pointer arithmetic, where any address can be known by knowing the base address. This is because every position of \text{pptr} (\textbf{pptr[0]}, \textbf{pptr[1]} ...) is allocated individually. These pointers do however point to contiguously allocated memeory. The three memory addresses are the same, for the same reasons as all of $0:$, they are equivalent statements: 0x7FB322401F50.
\begin{lstlisting}
    printf("2: *pptr + 1 = %p\n", *pptr + 1);
    printf("2: *(pptr+0) + 1 = %p\n", *(pptr + 0) + 1);
    printf("2: &pptr[0][1] = %p\n", &pptr[0][1] );  
\end{lstlisting}
\textbf{Explanation: } As stated in 0, \textbf{*(pptr + 0)} is equivalent to \textbf{*pptr}, thus the addresses for \textbf{*(pptr + 0) + 1} and \textbf{*pptr + 1} are the same: 0x7FB3224000E4. This is because \text{*pptr} is an integer pointer, and not a pointer pointer, and integers have a size of 4 bytes. Similar to before \textbf{pptr[0][1]} is equivalent to \textbf{*(*(pptr + 0) + 1)}, and \text{\&*(*(pptr + 0) + 1)} is the same as \textbf{*(pptr + 0) + 1}, and holds the same address as mentioned previously: 0x7FB3224000E4.
\begin{lstlisting}
    printf("3: *(pptr[1] + 1) = %d\n", *(pptr[1] + 1) );
    printf("3: *( *(pptr + 1) + 1) = %d\n", *( * (pptr + 1) + 1) );
    printf("3: pptr[1][1] = %d\n", pptr[1][1] );
\end{lstlisting}
\textbf{Explanation: }These statements are all equivalent forms of \textbf{pptr[1][1]} which is the integer contained at the memeory location \textbf{*(pptr + 1) + 1}, which is $5$. This does not need much more explanation, since these equivalences have been shown multiple times above.
\end{document}
