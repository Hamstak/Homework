\documentclass{article}
\usepackage{amsmath}
\usepackage{amssymb}
\usepackage{multicol}
\usepackage{amsfonts}
\usepackage{fancyhdr}
\usepackage{hyperref}
\usepackage{tabularx}
\usepackage[margin=1in]{geometry}
\usepackage{listings}
\usepackage{color}
\usepackage{graphicx}
\graphicspath{{/home/tim/Pictures/}}

\definecolor{dkgreen}{rgb}{0,0.6,0}
\definecolor{gray}{rgb}{0.5,0.5,0.5}
\definecolor{mauve}{rgb}{0.58,0,0.82}

\lstset{frame=tb,
  language=C,
  aboveskip=3mm,
  belowskip=3mm,
  showstringspaces=false,
  columns=flexible,
  basicstyle={\small\ttfamily},
  numbers=none,
  numberstyle=\tiny\color{gray},
  keywordstyle=\color{blue},
  commentstyle=\color{dkgreen},
  stringstyle=\color{mauve},
  breaklines=true,
  breakatwhitespace=true,
  tabsize=3
}
\begin{document}
\begin{flushright}
  Tim Tanasse\\
  CSCD 240\\
  \today \\
  Notes \# \# \\
\end{flushright}
\section*{Problem 1: Write, compile, and execute the following C file and answer the following questions}
\begin{itemize}
  \item \textbf{What is the size of ptr on cslinux?}\\
  \textbf{ptr} takes up 8 bytes
  \item \textbf{What is the size of arr[0] on cslinux?}\\
  \textbf{arr[0]} takes up 4 bytes
  \item \textbf{Draw a diagram of the memeory map of ptr and arr given that arr is at base address 0x5600bc}\\
  See last page
\end{itemize}
\section*{Problem 2: Take educated guesses about the functioning of some code involving pointers}
\begin{itemize}
  \item \begin{lstlisting}
    ptr++;
printf("*ptr %d\n", *ptr);
printf("ptr %p\n", ptr);
  \end{lstlisting}
  \textbf{Explanation:} The pointer is incremented, which moves it by the size of an integer (4) from 0x5600bc to 0x5600c0. Prints the value at the ptr location: 400, or \textbf{arr[1]}. Prints the memory address contents of \textbf{ptr}, which should have shifted 4 bytes from the previous address.
  \item \begin{lstlisting}
    *++ptr;
printf("*++ptr %d\n", *ptr);
printf("ptr %p\n", ptr);
  \end{lstlisting}
  \textbf{Explanation:} The unary operator \textbf{*} is read from right to left, which makes \textbf{*++ptr} execute similarly to \textbf{*(++ptr)}. The pointer is incremented from 0x5600c0 to 0x5600c4. Prints the value at the new memeory location: 600, or \textbf{arr[2]}. Prints the memory address contents of \textbf{ptr}, which should have shifted 4 bytes from the previous address.
  \item \begin{lstlisting}
    *ptr++;
printf("*ptr++ %d\n", *ptr);
printf("ptr %p\n", ptr);
  \end{lstlisting}
  \textbf{Explanation:} In the same way as the previous explanation, \textbf{*ptr++} is interpreted as \textbf{*(ptr++)} and the pointer \textbf{ptr} is incremented. Prints the value at the new memory location: 800, or \textbf{arr[3]}. Prints the memory address contents of \textbf{ptr}, which should have shifted 4 bytes from the previous address 0x5600c4 to 0x5600c8.
  \item \begin{lstlisting}
    ptr = arr; // reset ptr, no need to explain this statement

    // fun with printf repeat last couple of commands
    printf("*++ptr %d\n", *++ptr);
    printf("ptr %p\n", ptr);
  \end{lstlisting}
  \textbf{Explanation:} Due to the functionality of \textbf{++} and the previously stated functionality of \textbf{*}, the pointer is incremented before being used in \textbf{printf}. Prints the value that the pointer now points to: 400, or \textbf{arr[1]}. Prints the memeory address contents of \text{ptr}: 0x5600c0.
  \item \begin{lstlisting}
    printf("*ptr++ %d\n", *ptr++);
    printf("ptr %p\n", ptr);
  \end{lstlisting}
  \textbf{Explanation:} Unlike the above, post incrementing results in the pointer being incremented after it is used in \textbf{printf}. Prints the dereferenced pointer value: 400 or \textbf{arr[1]}, then increments the pointer. Prints the new memory address contents of \textbf{ptr}: 0x5600c4.
  \item\begin{lstlisting}
    ptr = arr; // reset ptr, no need to explain this statement
    *ptr += 1;
    printf("*ptr %d\n", *ptr);
    printf("ptr %p\n", ptr);
\end{lstlisting}
  \textbf{Explanation:} The pointer is dereferenced, and then the value is incremented by 1 from 200 to 201. Prints the derefenced value: 201, or \textbf{arr[0]}. Prints the memory address contents of \textbf{ptr}: 0x5600bc.
  \item\begin{lstlisting}
printf("*(ptr+1) = %d\n", *(ptr+1));
\end{lstlisting}
\textbf{Explanation} Takes the pointer value and increases it by 4 bytes. Prints the value at that adjusted location: 400.
\item \begin{lstlisting}
  ptr = arr; // reset ptr, no need to explain this statement
  *(arr+2) = *ptr+100;
  printf("*(arr+2) = %d\n", *(arr+2));
\end{lstlisting}
\textbf{Explanation:} The \textbf{arr} variable is a memory address of the variable \textbf{arr[0]}. So the value at \text{arr + 2} or \textbf{arr[2]} is getting the derefernced value of \textbf{ptr} + 100. Prints the new derefenced value: 301, or \text{arr[2]}
\item \begin{lstlisting}
ptr = arr + 5;
printf("*ptr %d\n", *ptr);
printf("ptr %p\n", ptr);
\end{lstlisting}
\textbf{Explanation:} The pointer is set to \text{arr + 5} which is equivalent to getting the memory address of \textbf{arr[5]}. Prints the dereferenced value: 1200. Prints the memeory address content of \textbf{ptr}: 0x5600D0
\item \begin{lstlisting}
  
ptr = arr; // reset ptr, no need to explain this statement

arr[2] = *(ptr + 5);
printf("arr[2] = %d\n", arr[2]);

\end{lstlisting}
\textbf{Explanation:} \textbf{arr[2]} gets the dereferenced value of \textbf{ptr + 5}, which is \textbf{arr[5]}. Prints the value of \textbf{arr[2]}: 1200.
\item \begin{lstlisting}
ptr = (arr + 10);
printf("ptr %p\n", ptr);
printf("*ptr %d\n", *ptr);
\end{lstlisting}
\textbf{Explanation:} \textbf{ptr} gets ten 4 byte increments, which puts it outside of the array \textbf{arr}. Prints the memory address content of \textbf{ptr}: 0x5600e4. Prints the dereferenced value at that memeory location.
\end{itemize}
\section*{Problem 3: Corrections}
\begin{itemize}
  \item \textbf{19:} Correct Guess
  \item \textbf{23:} Correct Guess
  \item \textbf{27:} Correct Guess
  \item \textbf{34:} Correct Guess
  \item \textbf{37:} Correct Guess
  \item \textbf{42:} Correct Guess
  \item \textbf{46:} Correct Guess
  \item \textbf{50:} Correct Guess
  \item \textbf{53:} Correct Guess
  \item \textbf{59:} Correct Guess
  \item \textbf{62:} Correct Guess
\end{itemize}
\pagebreak
\includegraphics[width=\textwidth]{lab6-1.png}\\
\includegraphics[width=\textwidth]{lab6-2.png}\\
\includegraphics[width=\textwidth]{lab6-3.png}
\pagebreak
\end{document}
