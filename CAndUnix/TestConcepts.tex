\documentclass{article}
\usepackage{amsmath}
\usepackage{amssymb}
\usepackage{multicol}
\usepackage{amsfonts}
\usepackage{fancyhdr}
\usepackage{hyperref}
\usepackage{tabularx}
\usepackage[margin=1in]{geometry}
\usepackage{listings}
\usepackage{color}

\definecolor{dkgreen}{rgb}{0,0.6,0}
\definecolor{gray}{rgb}{0.5,0.5,0.5}
\definecolor{mauve}{rgb}{0.58,0,0.82}

\lstset{frame=tb,
  language=Java,
  aboveskip=3mm,
  belowskip=3mm,
  showstringspaces=false,
  columns=flexible,
  basicstyle={\small\ttfamily},
  numbers=none,
  numberstyle=\tiny\color{gray},
  keywordstyle=\color{blue},
  commentstyle=\color{dkgreen},
  stringstyle=\color{mauve},
  breaklines=true,
  breakatwhitespace=true,
  tabsize=3
}
\begin{document}
\begin{flushright}
  Tim Tanasse\\
  Java 210\\
  \today \\
  Notes \# \# \\
\end{flushright}
\section*{Concepts for Test}
\subsection*{Unix}
\begin{itemize}
  \item Absolute Path
  \item Relative Path
  \item Shell
  \item Bash/csh/tcsh
  \item Command ls -al:
  \begin{itemize}
    \item Example:\\
    drwxr-xr-x, 3, will, finance, 4096, \today, will
    \item Meaning:\\
    \[\frac{\text{d}}{\text{type}}\cdot\frac{\text{rwxr-xr-x}}{\text{Permissions}}\cdot\frac{3}{\text{links}}\cdot\frac{\text{will}}{\text{Owner}}\cdot\frac{\text{finance}}{\text{Group}}\cdot\frac{\text{4096}}{\text{sizeBytes}}\cdot\frac{\text{\today}}{\text{Date last modified}}\cdot\frac{\text{will}}{\text{fileName/DirectoryName}}\]
  \end{itemize}
  \item permissions:\\
  'd' directory\\
  '-' ordinary file\\
  'l' symbolic link
  \item three access types
  \item user categories:
  \begin{itemize}
    \item user
    \item user group
    \item other users
    \item 'r', 'w', 'x'
  \end{itemize}
  \[\frac{\text{RWX}}{\text{User}}\cdot\frac{\text{R-X}}{\text{Group}}\cdot\frac{\text{R-X}}{\text{Other}}\]
  'r' - read, 'w' - write, 'x' - execute, '-' - no permission for field.
  \item '..' and '.' and other cd commands
  \begin{itemize}
    \item '..' - parent directory
    \item '.' - current directory
    \item example: cd ../../ -> parent parent
    \item '-' equivalent to 'back' brings to previous cd -option result
  \end{itemize}
  \item \textbf{rmdir} - removes empty directory
  \item \textbf{mkdir} - adds directory to path or current directory
  \item \textbf{cp} [Source File] destination - copy a file to a destination
  \begin{itemize}
    \item if the destination is a file, only one source file is allowed
    \item if the destination is a folder, many sources can be copied into a destination folder
    \item option -r (meaning recursively) copies all files and subfolders and so one to the destination.
    \item use of wildcard implies all of type
  \end{itemize}
  \item \textbf{mv} [Options] <source> <destination>
  \begin{itemize}
    \item move file or directory to another place
    \item optional rename a file or directory
    \item automatically recursive
  \end{itemize}
  \item \textbf{rm} [Options] <source>
  \begin{itemize}
    \item rm * is very dangerous
    \item rm is dangerous
    \item good practice \textbf{alias rm ='rm -i'} asks for confirmation before deletion
    \item -r against is recursive
    \item rm itself does not remove directories
  \end{itemize}
  \item \textbf{alias} [command] = [command];
  \begin{itemize}
    \item temporary
  \end{itemize}
  \item \textbf{touch} [Option] <file>
  \begin{itemize}
    \item Creates an empty file (if used without options, and file does not exist)
    \item adjust the timestamp of the specified file to current time (without options)
  \end{itemize}
\end{itemize}
\subsection*{Permissions}
\begin{itemize}
  \item Why is it advantagous?
  \begin{itemize}
    \item makes things safer from malicious files
  \end{itemize}
\end{itemize}
\subsection*{Other Commands}
\begin{itemize}
  \item \textbf{chmod} <mode> <file>
  \begin{itemize}
    \item Change file/directory permissions base on what is specified in the <mode> argument.
    \item The format of <mode> is a combination of 3 fields:
    \begin{itemize}
      \item Who is affected: [u]ser, [g]roup, [o]ther, [a]ll users
      \item Whether adding or removing permissions: + or -
      \item Which permissions are being added/removed [w]rite, [r]ead, e[x]ecute
      \item Example: \textbf{chmod ug+rw myfile}
      \end{itemize}
    \item \textbf{chmod} with numeric mode
    \begin{itemize}
      \item Uses three digits of octal numbers: 777, 755
      \item Ordered as ugo
      \item Sum of the permission values
      \item read: 4, write: 2, execute: 1
    \end{itemize}
  \end{itemize}
\end{itemize}
\subsection*{Read Files}
\begin{itemize}
  \item \textbf{nano} [Option] <file>
  \begin{itemize}
    \item Simple text editor in bash
  \end{itemize}
  \item \textbf{cat} <filename>
  \begin{itemize}
    \item Prints the contents of the file t othe terminal
    \item with two filenames it appears to concatinate the two files file 1 and file 2
  \end{itemize}
  \item \textbf{more} <filename>
  \item \textbf{less} <filename>
  \item \textbf{head} [option] <filename> read from beginning
  \item \textbf{tail} [option] <filename> read from end
  \begin{itemize}
    \item option is numlines
    \item default is ten
  \end{itemize}
  
\end{itemize}
\subsection*{Wild Cards}
Also called metacharacters
\begin{itemize}
  \item * - will match any string including null string
  \item ? - match a single character(cannot be null)
  \item [] - will match any one character inside the brackets, can use dash for a range of characters or numbers. Commas can separate different range references. Commas are not necessary between ranges, but does increase readability.
  \item \^ - used in [], matches any  single character inside the bracket. Essentially [not] logic.
\end{itemize}
\subsection*{PATH}
\begin{itemize}
  \item when searching for commands, will search all paths in the PATH variable
  \item echo \$PATH - shows the contents of the path variable
  \item export \$PATH="\$PATH":newPath adds newPath to the PATH variable
\end{itemize}
\subsection*{.bashrc}
Permanent location of PATH variable among other permantent bash things.
\begin{itemize}
  \item executed whenever a new shell is started
  \item Ti add a PATH entry permanently
  \begin{itemize}
    \item add a command to bashrc
  \end{itemize}
\end{itemize}
\subsection{Shell Sequence, Redirection, Piping}
\begin{itemize}
  \item When you type in a command, shell goes through quite a complicated sequence of operations to process your request.
  \begin{itemize}
    \item Check aliases
    \item parameter expansion, substitution, quotes removal
    \item Shell function
    \item built in command
    \item hash tables for previously executed commands
    \item PATH VARIABLES
  \end{itemize}
  If all of those fail, command will fail.
  \item \textbf{Find} advanced search, equivalent to advanced search in windows.
\end{itemize}
\subsection*{I/O Stream}
\begin{itemize}
  \item Every Unix program has three streams opened for it, one for input, one for outpot, one for errors.
  \begin{itemize}
    \item input stream is reffered to as standard input
    \item output stream is reffered to as standard output
    \item error stream is reffered to as standard error
  \end{itemize}
  \item stdin stdout stderr
  \item every device is considered a file
  \item The integer file discriptors associated with the streams stdin, stdout, and stderr are 0, 1, 2 respectively
  \item The standard output -- the screen on your terminal by default.
  \item \textbf{cat} reads from stdin by default (integer reference 0)
  \item you can redirect these streams to a file, or somewhere else in general
  \item You redirect these streams to a file, or even another command with redirection
  \item \textbf{wc -l < hw1.c} counts lines from file hw1.c with the -l option to count lines
  \item this is an example of input redirection
  \item \textbf{mail tony@good.org < todo.txt} redirects text from file to email.
  \item \textbf{>} redirects standard output to a file
  \begin{itemize}
    \item if you redirect the standard output to an existing file, it will overwrite the file, unless you use the \textbf{>>} operator
    \item \textbf{>>} appends the standard output to the contents of the existing file.
  \end{itemize}
  \item to redirect the standard error stream to a file, use the \textbf{>} opperator preceded by a 2
  \begin{itemize}
    \item 
  \end{itemize}
\end{itemize}
\subsection*{Piping}
\begin{itemize}
  \item Piping is when you connect the standard output of one command to the standard input of another
  \begin{itemize}
    \item \textbf{ls -l | wc -l}
    \item count how many files and folders in the current directory
    \item \textbf{who | grep "tony"}
    \item check whether tony is logged in.
    \item First run who to list all logged in users then search 'tony' in that list
    \item Differs from I/O redirection
  \end{itemize}
\end{itemize}
\subsection*{grep}
\begin{itemize}
  \item search command
  \begin{itemize}
    \item -i case insensitive
    \item -r recursive
    \item grep -rn dbConnect --include=*.c ./ searches all c files from current directory onwards that contain dbConnect
    \item -v invert, all lines that do not contain
    \item -c count instead of display lines
    \item -x provided pattern has to match a line exactly
    \item powerful with piping
    \item \textbf{ls -LR | grep errfile}
    \begin{itemize}
      \item -LR follows links and recursive gets all \textbf{ls} entries for all subdirectories and current directory
      \item pipes the output to grep and grep searchs for a file called errfile.
    \end{itemize}
      \item \textbf{tail -n8 a\_file | grep "boo"}
  \end{itemize}
\end{itemize}
\section*{Regular Expressions}
\begin{itemize}
  \item A test string of special characters that specifies a set of patterns to match
  \item Most characters represent themselves
\end{itemize}
\subsection*{REGEX metacharacters}
\begin{itemize}
  \item . - will match any one character with the exception of the newline character.
  \item '.wn' matches wn preceded by one character.
  \item * - matches the preceding character zero or more times
  \item '-*' matches '--', '----', '-', and so on.
  \item \$ - matches the end of the line. So 'a\$' matches 'a' only when it is the last character on a line
  \item \^ - matches the beginning of the line. so '\^a' matches 'a' only when it is the first character on a line.
  \item '\^\$' - matches empty line
  \item '\^.' - matches any non empty line
  \item '\^.*\$' - matches any line
  \item '\' - escape character, just like latex
  \item brackets function the same way they do for the shell 
\end{itemize}
\subsection*{Find command}
\begin{itemize}
  \item \textbf{find . -name ".txt" -exec wc -l \{\} ';'}
  \item search the current folder and all subfolders for the pattern, and then executes the command 'wc -l' with the replaced file name '\{\}' and ends the exec clause ';'
  \item man find gives more info on find
\end{itemize}
\subsection*{Echo}
\begin{itemize}
  \item Escape character \
  \item Single Quotes
  \begin{itemize}
    \item Variables banes are bit exoabded
    \item Everthing inside is treated literally
    \begin{itemize}
      \item NAME=john - here NAME is a shell variable
      \item echo '$NAME' -$NAME
    \end{itemize}
  \end{itemize}
  \item Double Quotes
  \begin{itemize}
    \item Variable names are expanded
    \item Special meaning of metacharacters are preserved
    \begin{itemize}
      \item NAME=john
      \item echo "\$NAME" - john
    \end{itemize}
    \item Preserves meaning of the following
    \begin{itemize}
      \item backquotes
      \begin{itemize}
        \item accent characters
        \item the output of the command generated by the command replaces the command itself
        \item echo `pwd` - /home/tim
        \item echo "The current time is `date`." - The current time is Wed Dec 25 20:23:55 PST 2013.
      \end{itemize}
      \item dollar sign - expand variable
      \item backslash related replacements
    \end{itemize}
  \end{itemize}
\end{itemize}
\subsection*{Link}
\begin{itemize}
  \item \textbf{ln}
  %\item \textbf{ln -s}$ \tilde $\textbf{/cscd240/mylab1.txt mylab\_link}
  \begin{itemize}
    \item -s means symbolic links or soft links
    \item Create a symbolic link mylab\_link, which links to file ~/cscd240/mylab1.txt
  \end{itemize}
  \item if you modify the 'symbolic link' you are modifying the actual file
  \item the symbolic link can exist on a different disk partition than the target
  \item directory also ca be linked with symbolic link
\end{itemize}
\subsection*{archive files: tar}
\begin{itemize}
  \item \textbf{tar -cvf myArch.tar file1 file2 file3 folder1}
  \begin{itemize}
    \item create an archive file myArach.tar, which contains two regular files file1, file2, and all contents in folder1.
    \item c means creating new archive file
    \item v means verbose output and listing all files in the archive file
    \item f means writing to the provided archive file name
  \end{itemize}
  \item \textbf{tar -xvf myArch.tar}
  \item the z option using zip style compression/decompression
\end{itemize}
\subsection{processes and jobs}
\begin{itemize}
  \item Process - a program in execution
  \begin{itemize}
    \item editing a file in nano
    \item listening to music with some program
  \end{itemize}
  \item A process consists of resources that the program needs to run
  \begin{itemize}
    \item Such as a program instructions, the files descriptors, the memory to be used, (Registers, Stack, Heap), IPC tools, etc.
  \end{itemize}
  \item The OS keeps track of processes by assigning each a number, the process id or PID.
  \item When you invoke a system utility or application program from a shell, one or mroe child processes are created by the shell
  \item An important process that is always present is the init process
  \item Show the states of the processes and their PID
  \begin{itemize}
    \item \textbf{ps}
    \begin{itemize}
      \item without any argument, it will list PIDs for processes owned by the current shell
    \end{itemize}
    \item \textbf{ps -aux}
    \begin{itemize}
      \item shows full information about all processes on BSD machine
      \item \textbf{ps -fae} - linux systems
    \end{itemize}
    \item \textbf{ps -u user}
    \begin{itemize}
      \item Shows all processes belonging to the user
    \end{itemize}
  \end{itemize
  %\item \textbf{kill [PID]} - sends different level signals to the running process, requesting it to shutdown}
  \item \textbf{kill -l - kist all signals}
  \item \textbf{kill -9 [PID] - kill abruptly (immeadiate), useful for stalls and looping issues}
\end{itemize}
\subsection*{Job}
\begin{itemize}
  \item Actually the same thing as a process but restricted to the processes started by a particular shell
  \begin{itemize}
    \item Attached t othat shell session
    \item if Shell closes, jobs in the shell terminate
  \end{itemize}
  \item Jobs can either be in the foreground or the background
  \begin{itemize}
    \item bg \%n runs the job in the background.
  \end{itemize}
  \item \textbf{job -l}
  \begin{itemize}
    \item List the active jobs and its pid
    \item \textbf{[1]+ 1490 Running find / -print 1>output2>errors &}
    \begin{itemize}
      \item [1] is the job number
      \item 1490 is the PID
      \item find / -print 1>output2>errors &
      \item start a job as background job usuing & at the end.
    \end{itemize}
    \item note that if you have more than one job you refer to the job as \%n where n is the job number
    \item \textbf{kill -9 \%1} kills the first job.
  \end{itemize}
  \item ctrnl + z, suspend or sleep the foreground, cntrl + c kills the foreground
\end{itemize}
\end{document}
