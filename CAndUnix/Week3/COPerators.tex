\documentclass{article}
\usepackage{amsmath}
\usepackage{amssymb}
\usepackage{multicol}
\usepackage{amsfonts}
\usepackage{fancyhdr}
\usepackage{hyperref}
\usepackage{tabularx}
\usepackage[margin=1in]{geometry}
\usepackage{listings}
\usepackage{color}

\definecolor{dkgreen}{rgb}{0,0.6,0}
\definecolor{gray}{rgb}{0.5,0.5,0.5}
\definecolor{mauve}{rgb}{0.58,0,0.82}

\lstset{frame=tb,
  language=C,
  aboveskip=3mm,
  belowskip=3mm,
  showstringspaces=false,
  columns=flexible,
  basicstyle={\small\ttfamily},
  numbers=none,
  numberstyle=\tiny\color{gray},
  keywordstyle=\color{blue},
  commentstyle=\color{dkgreen},
  stringstyle=\color{mauve},
  breaklines=true,
  breakatwhitespace=true,
  tabsize=3
}
\begin{document}
\begin{flushright}
  Tim Tanasse\\
  CSCD 240\\
  \today \\
  Notes \# \# \\
\end{flushright}
\section*{C Operators}
\begin{itemize}
  \item \#define or const to define a constant
  \item Basic athirmetic operators
  \item Exanded arirthmetic: \%, ++, --
  \item preincriment and postincriment are included
  \begin{itemize}
    \item a++ uses current value of a, then incriments the value of a
    \item ++a uses the new value of a, proceding the incriment.
  \end{itemize}
  \item Relational operators are the same as Java.
  \item Bitwise Operators
  \begin{itemize}
    \item \&
    \item Binary AND operator copies a 1-bit to the result if it exists in both operands
    \item |
    \item Binary OR operator copies a 1-bit to the result if it exists in either operands
    \item Up Carrot
    \item Binary XOR operator, copies a 1-bit to the result if it exists in only one operand
    \item Tilde
    \item Binary NOT operator, copies a 1-bit to the result if it does not exist in the operand.
    \item $<<$
    \item Binary Left Shift operator, the left operands value is moved left by the number of bits specificed in the right operand: $A<<2$
    \item Left Shift same as $*2^k$ where $k$ is the number of bits shifted
    \item $>>$
    \item Binary Right Shift, the left operands value is moved left by the number of bits specificed in the right operand: $A<<2$
    \item Right Shift same as$\frac{}{2^k}$ where $k$ is the number of bits shifted
    \item Assignment: $<<=$, $>>=$, $\&=$, $\text{carrot}=$, $\text{tilde}=$
  \end{itemize}
\end{itemize}
\end{document}
